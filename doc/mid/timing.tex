\section{Timing}
\subsection{Profiling}
In order to find the bottleneck of the code, we first profiled
 timing of each steps using Intel’s VTune Amplifier. See ~\ref{lst:amplxe_command}.
 
\begin{lstlisting}[caption={VTune Amplifier Command},label={lst:amplxe_command},frame=single,language=bash]
amplxe-cl -collect advanced-hotspots ./shallow
amplxe-cl -report hotspots -source-object function=<NAME>

amplxe-cl -report hotspots -r r001ah/ > all
amplxe-cl -report hotspots -source-object function="Central2D
<Shallow2D, MinMod<float>>::compute_step" > compute_step
\end{lstlisting}

\subsection{Initial Profile Result}

From the code~\ref{lst:initial_profile_result}, we could figure out "limited\_derivs", "compute\_step", and "compute\_fg\_sppeds" take longest time (See profile.sh code). Among these results, "limited\_derivs" function definitely was the worst bottleneck. 

\begin{lstlisting}[caption={Initial Profile Result},label={lst:initial_profile_result},frame=single,language=bash]
Function                     Module        CPU Time
-----------------------------------------  --------
limited_derivs               shallow         1.378s
compute_step                 shallow         0.640s
compute_fg_speeds            shallow         0.219s
_IO_fwrite                   libc-2.12.so    0.019s
_IO_file_xsputn              libc-2.12.so    0.015s
[Outside any known module]   [Unknown]       0.014s
run                          shallow         0.006s
write_frame                  shallow         0.006s
solution_check               shallow         0.004s
offset                       shallow         0.002s
do_lookup_x                  ld-2.12.so      0.001s
operator[]                   shallow         0.001s
\end{lstlisting}

\subsection{Initial Timing Result}
We made simple script to generate timing plots (See plotter.sh code). Here, x-axis indicates the number of cells per side. We swept this from 100 to 1000. Y-axis shows the number of cells per side per average time, which means cell per seconds. Initial timing result can be found at Figure ~\ref{fig:initial_timing_result}.

\begin{figure}[h]
    \centering
    \includegraphics[width=0.8\textwidth]{figs/init.pdf}
    \caption{Initial Timing Result}
    \label{fig:initial_timing_result}
\end{figure}