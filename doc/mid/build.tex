\section{Improved Build System}\label{sec:build}
The development of an efficient shallow water simulator is a complicated task
that involves a lot of rapid prototyping, and the efficiency with which we can
develop simulators is limited in part by the overhead of creating, modifying,
and running simulators. In order to reduce this overhead, we improved and
streamlined the build system.

The original \ttt{water} build system did not accommodate an easy way to run
multiple versions of a shallow water simulator. Instead, we were forced to
modify the type definitions at the top of \ttt{driver.cc} before compilation
which is slow and error-prone. To alleviate this burden, we modified
\ttt{driver.cc} to instantiate one of a set of simulators based on a macro
which is provided at the command line during compilation. For example, assume
we have two simulators \ttt{SimA} and \ttt{SimB}. To build \ttt{driver.cc} to
use \ttt{SimA}, we run a command similar to \ttt{icpc -DVERSION\_a -o
shallow\_a driver.cc}. Similarly, to build \ttt{driver.cc} to use \ttt{SimB},
we run something similar to \ttt{icpc -DVERSION\_b -o shallow\_b driver.cc}.
Moreover, we abstract these commands using the Makefile so that we only have to
run \ttt{make shallow\_a} or \ttt{make shallow\_b}.

This improved build system makes it much easier to develop and evaluate many
versions of simulators simultaneously.
