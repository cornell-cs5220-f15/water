\section{Parallelization using OpenMP}\label{sec:parallelization}

In the initial stage of the project we implemented OpenMP to parallelize our code in hopes of improving the computational performance. OpenMP stands for Open Multi-threads Programming; it is an industry standard API of C/C++ for shared memory parallel programming.  The way OpenMP works is first decomposing the work into smaller chunks, and then assigning the tasks to different threads such that multiple threads can share work in parallel. When the work is done, the threads will synchronize implicitly by reading and writing shared variables.

Although OpenMP is a very powerful tool to use to increase speedup, we found that using OpenMP alone without any blocking or vectorization actually worsen the performance. Two of the reasons were load imbalance overhead and Parallel overhead. Load imbalance overhead is when the threads are performing unequal amount of work in the shared region. The faster thread will need to wait for all the other threads to finish the work before they can synchronize the information; when the threads are not doing any work/idling, it accumulates synchronization overhead.

Parallel overhead is the accumulated time that takes to start threads, distribute tasks to threads, and etc. Using Vtune to analyze the runtime of our codes, it appears that the time for the processor to compute information in most of our for-loops was in the range of micro-seconds. As a result, using \ttt{\#pragma omp parallel for} would not be beneficial because it takes much more time to distribute work to threads than to compute the code in serial.

However, after blocking(details in section TODO) was implemented, we were able to optimize the performance by about 6X using \ttt{\#pragma omp parallel for collapse(2)} clause as shown in Figure ~\ref{fig:omp}. The \ttt{run\_block} funciton decomposed the block into smaller domains with a ring of ghost cells wrapped around each domain(more details in section TODO) \ttt{\#pragma omp parallel for collapse(2)} command solved the load imbalance problem by increasing the total number of iterations that will be partitioned across the available number of OMP threads and hence reduced the granularity of work to be done by each thread.

\begin{figure}[h]
\centering
\begin{CPP}[firstnumber=476]
 #pragma omp parallel for collapse(2)
      for (int by = 0; by < BY; ++by) {
         for (int bx = 0; bx < BX; ++bx) {
              run_block(io,dt,bx,by);
                 }
           }
\end{CPP}
\caption{A loop from \ttt{central2d\_opt.h}.}
\label{fig:omp}
\end{figure}
