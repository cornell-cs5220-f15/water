\section{Future Work}\label{sec:future}
There are many optimizations that we plan on implementing before the final
submission.

\begin{itemize}
  \item \textbf{Vectorization.}
    We will complete the vectorization of our vectorized simulator.

  \item \textbf{Work Minimization.}
    The reference simulator performs some redundant computation. For example,
    the function \ttt{compute\_fg\_speeds} computes the flux for every ghost
    cell and its corresponding canonical cell. The number of ghost cells is
    small, but there may still be some benefits to removing this redundancy.

  \item \textbf{Compile Time Sizing.}
    The compiler is unable to vectorize certain loops because the bounds on the
    loops are unknown at compile time. Many of the loop bounds involve
    \ttt{nx}, \ttt{ny}, \ttt{nx\_all}, or \ttt{ny\_all} which are provided to
    the simulator at compile time. We plan on converting our simulators to take
    in these values as compile time template parameters. This will allow the
    compiler to perform more aggressive optimizations.

  \item \textbf{Blocking.}
    We plan on performing some sort of blocking to improve the memory locality
    of the simulator.

  \item \textbf{Ghost Celling and Domain Decomposing.}
    The reference simulator doesn't have much opportunity for naive
    parallelization. However, after we block our computation and introduce
    ghost cells between blocks, we will be able to parallelize our code at a
    much coarser level.
\end{itemize}
