\section{Verification}\label{sec:verification}
Shallow water simulation involves a lot of numerical computation, and
optimizing the simulation involves modifying and rearranging the code that
performs these computations in complex and intricate ways. Given the subtlety
and intricacy of the code, it is very easy to accidentally introduce bugs into
optimized code. The released code includes a \ttt{solution\_check} function
which verifies certain invariants of the simulation. For example, it checks
that the volume of water is conserved and that there are no negative water
heights. This invariant checking catches \emph{some} but not \emph{all} bugs.

In order to catch all bugs and guarantee the correctness of our code, we
developed our own simple verification system. In order to verify the
correctness of a simulator, we check its output against the reference simulator
provided in the release code. More concretely, assume we want to check the
correctness of an optimized \ttt{OptSim} simulator. We instantiate both
\ttt{OptSim} and the reference simulator \ttt{RefSim}.

\begin{CPP}
typedef Central2D   <Shallow2D,    MinMod   <Shallow2D   ::real>> RefSim;
typedef Central2DOpt<Shallow2DOpt, MinModOpt<Shallow2DOpt::real>> OptSim;
RefSim ref_sim;
OptSim opt_sim;
\end{CPP}

Then, we simulate \ttt{opt\_sim} and \ttt{ref\_sim} in lockstep and use a
function \ttt{validate} to check that the two simulators have identical grids
up to rounding error.

\begin{CPP}
for (int i = 0; i < frames; ++i) {
    opt_sim.run(ftime);
    ref_sim.run(ftime);
    validate(ref_sim, sim);
}
\end{CPP}

To ensure our verification system is itself correct, we developed an
intentionally buggy \ttt{BuggySim} simulator and a copy of the reference
simulator \ttt{CopySim}. Our system correctly reports that \ttt{BuggySim} is
incorrect while \ttt{CopySim} is correct.
