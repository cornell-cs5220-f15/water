\section{Domain Decomposition}\label{sec:domaindecomp}
Our shallow water simulators represent a region of water as a two-dimensional
grid of water heights. The simulation proceeds in several steps, and during
each step, values from one grid are transformed into values of another grid.
Symbolically, if one step of the simulation transforms values from a grid $A$
to a grid $B$ using a function $f$, then the following formula is used to
populate $B$.
\[
  \forall i, j.\, B_{ij} = f(A_{ij})
\]

A naive implementation of this pattern iterates over all values of $i$ and $j$
transforming values from $A$ to $B$. This implementation is analogous to a
naive implementation of matrix multiplication and suffers from the same
problems. More specifically, the naive implementation has poor cache locality
leading to poor performance on large grids.

To improve cache locality, we implemented blocking, or domain decomposition, in
which we operate on grids one block at a time. Domain decomposition not only
increases cache locality; it also allows for very easy parallelization. This is
discussed in detail in \secref{parallelization}.

The details of our implementation are not discussed here; they can be found in
\ttt{central2d\_block.h}.
