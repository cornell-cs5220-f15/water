%=========================================================================
% fig-parallel-node-results.tex
%=========================================================================

\begin{figure}[h]

  \begin{minipage}[t]{0.48\tw}
  \begin{subfigure}{\tw}

  \centering
  \includegraphics[width=1.0\tw]{fig-parallel-node-strong-results.py.pdf}
  \caption{\textbf{Strong Scaling Study}}
  \label{fig-parallel-node-strong-results}

  \end{subfigure}
  \end{minipage}%
  \hfill%
  \begin{minipage}[t]{0.48\tw}
  \begin{subfigure}{\tw}

  \centering
  \includegraphics[width=1.0\tw]{fig-parallel-node-weak-results.py.pdf}
  \caption{\textbf{Weak Scaling Study}}
  \label{fig-parallel-node-weak-results}

  \end{subfigure}
  \end{minipage}%

  \caption{\textbf{Performance Results of Parallel Implementation of
      Shallow Water Equation Solver Running on Compute Nodes --}
    Performance of the parallel implementation of the shallow water
    equation solver running on the Totient compute nodes compared against
    the provided serial implementation for all initial conditions. All
    speedups are the execution time normalized to the serial
    implementation. For the strong scaling experiment, we use the default
    200x200 grid.  For the weak scaling study, we increase the problem
    size at the same factor as the increasing number of threads (e.g.,
    200x200 grid for 1 thread, 400x400 grid for 4 threads, etc.). }

  \label{fig-parallel-node-results}

\end{figure}