%=========================================================================
% sec-parallel-device
%=========================================================================

\subsection{Parallelizing for the Accelerator Board}
\label{sec-parallel-device}

There are several ways we can leverage the Intel Xeon Phi accelerator
boards to further increase the speedups we see with the parallelized
implementation for the compute nodes. One approach is to execute the code
natively on the accelerator itself with no fine-grain
offloading. Although this approach makes it easier to tailor the code for
the accelerator and reduces overheads of copying data between the host
and the device memories, we are forced to run all the computation from
start to end on the accelerator. Another approach is to only offload
certain sections of the computation to the accelerator. With this
approach, there is a greater burden on the programmer to identify and
parallelize the most compute-intensive section of the code, but allows
him or her more flexibility in choosing which sections to accelerate.

As a starting point, we chose the second approach of offloading specific
sections of the computation to the accelerator and build off of the
parallel implementation for the compute nodes discussed in
Section~\ref{sec-parallel-node}. In this naive first-pass, we offload the
parallel section of the code and pass in the dimensions of the global
grid and the blocks, as well as a pointer to the global grid
itself. Inside the offloaded kernel, we still spawn off the specified
number of threads normally, except that instead of using pre-allocated
member \texttt{LocalState} objects, each thread creates its own
\texttt{LocalState} object on the stack. The \texttt{copy\_to\_local()} and
\texttt{copy\_from\_local()} functions were modified to copy the
flattened elements of the global grid passed in by the host to the local
grid vectors. Once the copy is complete, the other functions in the
offloaded kernel can take a pointer to the per-thread \texttt{LocalState}
object as an argument and access the vectors in this object similar to
before.  Because this naive implementation requires us to offload
computation before every timestep, there is an unnecessarily high
overhead of transferring data between the host and device memories.

We continued to optimize the parallel implementation for the accelerator
by offloading the entire \texttt{run()} function of the simulator. In
this approach, one main thread on the accelerator is responsible for
running the \texttt{apply\_periodic()} and \texttt{compute\_wave\_speeds()}
functions before parallelizing computation across multiple threads on the
accelerator for the other functions mentioned above. The benefit of this
approach is that we do not have to incur the overhead of copying data
between the host and the accelerator memories for offloading the
computation kernel for \emph{every} timestep. Instead, we only pay this
overhead once per frame in order to keep data local to the accelerator
for as long as possible. Note that here we only need to copy the global
grid vectors (i.e., \texttt{u\_offload}) and the simulator parameters
(e.g., \texttt{nx}, \texttt{nxblocks}, etc.) during the
offload. Functions meant to be called from the accelerator are annotated
with the \texttt{\_\_declspec(target(mic))} attribute and must
essentially be pure functions. In order to make it easier to pass in
simulator parameters to these functions that normally rely on the
simulator's member variables, we encapsulate the parameters copied to the
accelerator in a separate \texttt{Parameters} class.

%=========================================================================
% fig-parallel-node-results.tex
%=========================================================================

\begin{figure}[h]

  \begin{minipage}[t]{0.48\tw}
    \begin{subfigure}{\tw}

    \centering
    \includegraphics[width=1.0\tw]{fig-parallel-device-strong-results.py.pdf}
    \caption{\textbf{Strong Scaling Study}}
    \label{fig-parallel-device-strong-results}

    \end{subfigure}
  \end{minipage}%
  \hfill%
  \begin{minipage}[t]{0.48\tw}
  \begin{subfigure}{\tw}

  \centering
  \includegraphics[width=1.0\tw]{fig-parallel-device-weak-results.py.pdf}
  \caption{\textbf{Weak Scaling Study}}
  \label{fig-parallel-device-weak-results}

  \end{subfigure}
  \end{minipage}%

  \caption{\textbf{Performance Results of Parallel Implementation of
      Shallow Water Equation Solver Running on the Phi Coprocessors --}
    Performance of the parallel implementation of the shallow water
    equation solver running on the Totient compute nodes compared against
    the base serial implementation for all initial conditions. All
    speedups are the execution time normalized to the serial
    implementation. For the strong scaling experiment, we use an 800x800
    grid.  For the weak scaling study, we start the baseline at 200x200
    and increase the problem size at the same factor as the increasing number
    of threads as done with the compute node study, with the caveat that
    we multiplied the thread count by two to account for the roughly half
    clock speed of the Phi (e.g., 200x200 grid for 1 thread, 400x400 grid for
    8 threads, etc.). This was done because running so few threads on the Phi
    is really not worth investigating at all.}

  \label{fig-parallel-device-results}

\end{figure}



As the results show in Figure 4, although we were successful in running the algorithm
correctly on the Phi's, we hit a large bottleneck with respect to transferring memory.
The entire global \texttt{u} solution values must be copied over and back in each
frame (back so that the output can be written to disk).  Two very important components
of the Phi programming model are

\begin{enumerate}[1.]
	\item Transfer as little memory as possible,
	\item Attempt to do all allocations on the Phi at one time, and as infrequently as possible.
\end{enumerate}

So while transferring the global \texttt{u} values back and forth is not avoidable, an elegant
approach to avoiding allocation on the Phi when offloading within a loop was learned from the
tutorial here\\\\\centerline{https://software.intel.com/en-us/videos/part-2-efficient-parallel-programming-of-intel-xeon-processors-and-intel-xeon-phi}\\\\ 
is to \texttt{alloc\_if( < first iteration > )} \emph{only}, and \texttt{free\_if( < last iteration > )} \emph{only}.
Apparently, if you don't also change the \texttt{length(...)} specifier to be 0 you will still transfer things.
The point being that data re-use is fundamentally important, and we could not get it functioning.
This is largely because we are severely limited by C++ in the sense that classes and structs with things
such as dynamically allocated arrays like \texttt{std::vector} are very difficult to serialize.  There is a
branch on the repository called \emph{may\_as\_well} that attempts this, but it segfaults.  The most useful
lesson from that branch is that if you have a bunch of different \texttt{std::vector}'s, you can combine
them into one.

\noindent For example, instead of the current \texttt{local\_state.h} you can redefine internal fields and leave all the indexing functions the same
by doing
{\tiny
\begin{lstlisting}
    LocalState(int nx, int ny)
        : nx(nx), ny(ny),
          size(nx*ny),
          serial(
              (size) +// u_ // 0
              (size) +// v_ // 1
              ...
              (size) +// fx_// 6
              (size)  // gy_// 7
          ){
        vec *start = serial.data();
        u_         = start + (0*size);
        v_         = start + (1*size);
        ...
        fx_        = start + (6*size);
        gy_        = start + (7*size);
    }

    // Array accessor functions
    inline vec& u(int ix, int iy)  { return *(u_+offset(ix,iy));  }
    inline vec& v(int ix, int iy)  { return *(v_+offset(ix,iy));  }
    ...
    inline vec& fx(int ix, int iy) { return *(fx_+offset(ix,iy)); }
    inline vec& gy(int ix, int iy) { return *(gy_+offset(ix,iy)); }

    ...
    // danger: don't touch these if you don't know what you are doing...
    int nx, ny, size;
    aligned_vector serial;

    // Helper to calculate 1D offset from 2D coordinates
    inline int offset(int ix, int iy) const { return iy*nx+ix; }

    vec *u_;  // Solution values
    vec *v_;  // Solution values at next step
    ...
    vec *fx_; // x differences of f
    vec *gy_; // y differences of g
\end{lstlisting}
}

This makes serialization much more feasible.  Alas, the real solution here is to avoid working with C++ for Phi code
altogether if possible, which is rather frustrating.  They are not bitwise serializable (or at least not trivially so)
meaning direct transfer to the Phi is not possible.  Furthermore, after playing around with aligned allocators for the
\texttt{std::vector}, we discovered that the Phi compiler is rather outdated -- it is using g++ 4.7.0, which has a subtle
but significant impact on the way custom allocators must be written.  We were unable to make our custom allocator
backwards compatible.

%=========================================================================
% fig-parallel-node-results.tex
%=========================================================================

\begin{figure}[h]
    \centering
    \includegraphics[width=0.48\tw]{fig-parallel-device-batching-strong-results.py.pdf}
    \caption{\textbf{Strong Scaling Study}}
    \label{fig-parallel-device-batching-strong-results}
  \caption{\textbf{Performance Results of Parallel Implementation of
      Shallow Water Equation Solver Running on the Phi Coprocessors Using Batching --}
    Performance of the parallel implementation of the shallow water
    equation solver running on the Totient compute nodes compared against
    the tuned serial implementation for all initial conditions. All
    speedups are the execution time normalized to the serial
    implementation. We use an 800x800 grid.}

  \label{fig-parallel-device-batching-results}

\end{figure}



